\section{Mantener información de vuelos actualizada}
\subsection{Introducción}
La información sobre los vuelos con que cuenta el aeropuerto es la que puede  obtenerse de los sistemas de información de las aerolineas, de los mensajes provenientes de los transceptores, y la que eventualmente recaben los operadores del aeropuerto.
En el primer caso, el sistema informático del aeropuerto obtiene información relativa a los vuelos programados, es decir, horarios de arribo y de partida, y alteraciones tales como adelantos, retrasos y cancelaciones. También obtenemos de las aerolíneas los datos de los pasajeros que abordaron los aviones que parten, incluyendo su nacionalidad y los familiares que deban  contactarse para cumplir con el mandato de la Ley 10.302.

Los mensajes provenientes de los transceptores son procesados en el sistema y se utilizan para registrar la posición actual de los aviones en el aire que  tienen como destino el aeropuerto. 

Por último, el personal del aeropuerto ingresará al sistema en forma manual información de los vuelos ante eventualidades tales como falta de conectividad con las aerolíneas.

\subsection{Presunciones}
Continuando con lo descrito en la introducción, el sistema del aeropuerto para poder ofrecer buena calidad de información necesita que las aerolíneas mantengan actualizada la información de sus vuelos y pasajeros, y que esta última pueda ser accedida y recuperada por el sistema del aeropuerto de manera
on-line. 

\subsection{Diagrama de contexto}
\includegraphics[scale=0.40]{../../generated-src/actualizacion-contexto.png}

\subsection{Diagrama de objetivos}
\includegraphics[scale=0.35, angle=270]{../../generated-src/actualizacion-objetivos.png}

