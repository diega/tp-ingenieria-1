\section{Enunciado del trabajo práctico}
\subsection{Contexto}
El Aeropuerto Central de Camboya (ACC) está en planes de modernizar sus sistemas. Como parte de este plan, se decidió automatizar los sistemas correspondientes a la notificación de estados de los vuelos.  Es decir, el sistema que notifica el atraso, adelanto o cancelación de vuelos, tanto por partir como los que ya están volando.

Hoy en día esta tarea se realiza de forma manual, con el manejo de estado de los vuelos por partir y aterrizados consultando a las aerolíneas, y el manejo de los vuelos en el aire por medio de las comunicaciones con la torre de control.

Para el monitoreo de los aviones en el aire se hará uso de los mecanismos de radiofaro ya existentes en el aeropuerto, que reciben las señales de los aviones generadas por sus transceptores de a bordo. Estos dispositivos, además de recibir las señalizaciones del radiofaro para ubicar el aeropuerto, transmiten también su posición, altura y velocidad. Esto lo hacen de a períodos definidos por cada fabricante y/o actualizados por cada aerolínea que posee el avión. Sin embargo, son muchísimo más frecuentes que las comunicaciones entre la tripulación y la torre de control (el período en cuestión puede abarcar desde unos cuantos segundos a unos 10 minutos). Sin embargo, a diferencia de las comunicaciones entre la torre de control y los aviones, las comunicaciones por medio del transceptor son abiertas: cualquier persona con un receptor de radio puede captarlas y, también, generar información en el mismo canal con un transmisor. Por esto, los fabricantes de transceptores (son los mismos fabricantes del avión), emplean un método de ``\emph{security by obscurity}'' para mejorar la seguridad de la comunicación. Esto funciona de la siguiente manera: el formato de los datos transmitidos por los aviones consiste del nombre de la compañía fabricante (Boeing, por ejemplo), su matricula (LZR567, por ejemplo), su posición global (latitud/longitud), su altitud en metros y su velocidad de aire en km/h. La cuestión es que tras estos datos se adjunta un \emph{hash} de la información precedente. Dicho \emph{hash} actúa como verificador de que quien envía los datos es realmente un avión del fabricante antedicho. Para que los agentes confiados por la empresa verifiquen la veracidad de estos datos, cada fabricante distribuye en forma segura el algoritmo de generación de \emph{hash}. Este algoritmo suele ser actualizado cada día, siendo enviado automáticamente a todos los aviones y publicando por RSS la disponibilidad de una nueva versión para que el resto de los agentes la descarguen. El RSS incluye la dirección del servidor (IP y puerto) de donde descargar la nueva versión, donde la conexión debe hacerse por medio de un usuario y \emph{password} asignados por el fabricante a cada operador. El fabricante sólo garantiza que, tras 5 minutos de estar disponible el \emph{update}, todos los aviones en el mundo tendrán ya la versión operativa.

Para la recepción y traducción de los datos de los transceptores, ya está desarrollado un componente (Multiplexor de Información de Navegación Global de Aviones - \textsc{MINGA}) que escucha constantemente el canal de los transceptores y hace \emph{broadcast} de estos datos en la red (en principio se está utilizando el puerto 45678 para esto, pero puede modificarse en el futuro). Este componente no discrimina ningún avión (puede estar dando datos de navegación de aviones que no tienen relevancia ni de cerca para este aeropuerto), y no realiza ningún tipo de validación respecto de la autenticidad de estos datos.

El monitoreo de datos de los vuelos en tierra (aquellos por partir o los que ya han aterrizado), la administración del ACC tiene planeado mejorar el sistema manual actual reemplazándolo por otro que no dependa tanto en errores humanos introducidos en la carga de datos. La forma en la cual este nuevo sistema debería llevarse a la práctica no les queda del todo clara.

Toda la información obtenida, de todos los vuelos (por despegar, en el aire y aterrizados) se refleja en varios medios: en principio, en el servidor de datos del aeropuerto mismo, lo que sirve a los operadores de información. Además, la información se transmite también a las terminales de información (carteles y monitores en los pasillos) del aeropuerto y al sitio de Internet del aeropuerto. También se estudia la posibilidad de que cualquier persona pueda suscribirse a un servicio de información de un vuelo dado por Internet y reciba esta información en algún medio como email, \emph{Facebook} o SMS.

Por otra parte, entraron en vigencia dos nuevas leyes vinculadas a la aeronavegación aprobadas por la Honorable Cámara de Representantes Camboyanos:
\begin{description}
 \item[Ley 10.238:] Ningún habitante camboyano puede salir del país por vía aérea llevando consigo más de el equivalente a u\$s 1000 (mil dólares estadounidenses) en bienes tales como electrónica, efectivo, cheques, metales y piedras preciosas, obras de arte, etc. Esta restricción no afecta a diplomáticos camboyanos que se encuentren viajando por motivos oficiales. 
 \item[Ley 10.302:] Ante la eventualidad de que un vuelo saliente del ACC sufra un accidente aéreo, los familiares de aquellas personas que iban a bordo deben ser notificadas en un lapso no mayor a los 120 minutos desde el momento en el cual se suscite el siniestro. En el caso de haber extranjeros entre los viajeros la notificación debe ser enviada a las embajadas correspondientes.
\end{description}
La interpretación que tienen los camaristas camboyanos hace recaer la responsabilidad del cumplimiento de estas dos leyes sobre la administración del ACC.

\subsection{Entregables}
En esta primera etapa, la administración del Aeropuerto Central de Camboya quiere que le proporcionemos la mayor cantidad de información posible respecto de cómo nuestro sistema los ayudará a cumplir con sus objetivos (leyes, mayor confiabilidad en su información, etc.). También les interesa que les brindemos un análisis del \emph{scope} del \emph{software}, de forma tal de que puedan identificar claramente cómo se relacionará con los componentes actuales. Por último, nos piden que les otorguemos una serie de escenarios informales donde se ejemplifiquen situaciones hipotéticas (pero representativas) de funcionamiento esperado.
\newpage
